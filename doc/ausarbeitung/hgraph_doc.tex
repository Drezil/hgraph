\documentclass[a4paper]{scrartcl}
\usepackage{titling}

% LANGUAGE and LOCALE
\usepackage[USenglish,ngerman]{babel} % Deutsches Wörterbuch usw.
\newcommand{\en}[1]{\selectlanguage{USenglish}#1\selectlanguage{ngerman}}

% FONT
\usepackage[T1]{fontenc}
\usepackage[utf8]{inputenc}
\usepackage[babel=true]{microtype} % after babel
\usepackage{lmodern}
%\usepackage{times}
\makeatletter
\g@addto@macro\@verbatim{\microtypesetup{activate=false}}
\makeatother

% ENUMERATE and ITEMIZE
\usepackage{enumerate}
\usepackage{paralist}

% SCIENCE and MATH
\usepackage{amsmath}
\usepackage{amssymb}
\usepackage{icomma}
%\usepackage{units}
\usepackage{xfrac} % Brüche im Stil von ½ (\sfrac)

% PROGRAMMING
\usepackage{algorithm}
\usepackage[noend]{algpseudocode}

% TABLES
%\usepackage{array}
%\usepackage{multicol}	
%\usepackage{dcolumn}
%\usepackage{slashbox} % diagonale Linie in Tabellenzelle
%\usepackage{tabularx}

% LINKS and REFS
\usepackage[obeyspaces]{url}
\usepackage{hyperref}

% BIBLIOGRAPHY
\usepackage[style=german,german=quotes]{csquotes}
\usepackage[style=alphabetic,backend=biber,babel=hyphen,maxbibnames=10]{biblatex}
%\usepackage[authordate,backend=biber,babel=hyphen]{biblatex-chicago} 
\bibliography{hgraph.bib}

% FORMATING
\usepackage{anysize} % Makros zum Einstellen der Seitenränder
\parindent0em % Keine amerikanische Einrückung am Anfang von Paragraphen
\clubpenalty = 10000  % Schusterungen bestrafen
\widowpenalty = 10000 % Hurenkinder bestrafen
\displaywidowpenalty = 10000

% math commands
\newcommand{\trans}[1]{{#1}^\intercal}
\newcommand{\abs}[1]{\left\vert #1 \right\vert}
\newcommand{\card}[1]{\left\vert #1 \right\vert} % cardinality
\newcommand{\norm}[1]{\left\Vert #1 \right\Vert}

\newcommand{\setR}{\mathbb{R}}
\newcommand{\setN}{\mathbb{N}}

\DeclareMathOperator*{\argmin}{\arg\min}
\DeclareMathOperator*{\argmax}{\arg\max}

\hyphenation{IEEE} % hyphenation of default language
\begin{hyphenrules}{USenglish} % hyphenation of secondary language
  \hyphenation{IEEE}
\end{hyphenrules}

\begin{document}

\title{\en{Densely Connected Biclusters}}
\author{Stefan Dresselhaus \and Thomas Pajenkamp}
\date{\today} % TODO: Abgabedatum
\newcommand{\thesemester}{Wintersemester 2013/14}
\newcommand{\thecategory}{Programmierprojekt}
\newcommand{\thecourse}{Parallele Algorithmen und Datenverarbeitung}


%
% Deckblatt
%
\begin{titlepage}
  \begin{center} 
    \mbox{}
     \vspace{1cm}\\
     {\Huge \textbf{\thetitle}} \\[3em]
     {\huge \theauthor} \\[4em]
     {\Large \thecourse}\\[1em]
     {\Large \thecategory{} im \thesemester}\\[1em]
     
     \vspace{3cm}
     
     {\LARGE Universität Bielefeld -- Technische Fakultät}\\[2em]
     {\large \thedate}
  \end{center}
\end{titlepage}
%
% Inhaltsverzeichnis
%
%\setcounter{page}{1}
\microtypesetup{protrusion=false}
\tableofcontents
\microtypesetup{protrusion=true}

%\newpage

\section{Zielsetzung des Projekts}

Im Rahmen dieses Programmierprojekts wurde ein Programm entworfen und entwickelt, um \en{Densely Connected Biclusters}, im weiteren DCB, in einem biologischen Netzwerk zu ermitteln. Bei DCB handelt es sich um Teilgraphen eines Netzwerks, dessen Knoten untereinander hoch vernetzt sind und Objekte mit ähnlichen Eigenschaften repräsentieren. \par
Die Suche nach DCB ist ein NP-schweres Problem~\cite{Dummy}. Da mit einem geeigneten Algorithmus jedoch voneinander unabhängige Lösungspfade einzeln verfolgt werden können, ist das Problem gut für eine parallele Berechnung geeignet, wodurch die Gesamtlaufzeit stark reduziert werden kann.\par
%TODO: Referenz NP-schwer

%TODO ein bisschen biologische Motivation?

\en{
\subsection{Densely Connected Biclusters}
}

Ausgangsbasis ist ein ungerichteter ungewichteter Graph $G = (V, E)$, dessen Knoten mit $p$ Attributen versehen sind. Jedem Knoten $n$ ist zu jedem Attribut $i$ ein numerischer Wert $a_{ni}$ zugewiesen. \par
Ein DCB $D_k = (V_k, E_k)$ ist ein Teilgraph von $G$, der durch die Paramter $\alpha \in [0, 1]$, $\delta \in \setN$ und $\omega \in \setR^p$ beschränkt wird und die folgende drei Eigenschaften erfüllt.
\begin{itemize}
\item Der Teilgraph ist zusammenhängend.
\item Die Dichte des Teilgraphen unterschreitet einen Schwellenwert $\alpha$ nicht, also $\frac{2 \cdot \card{E_k}}{\card{V_k}(\card{V_k}-1)} \geq \alpha$.
\item Für mindestens $\delta$ Attribute liegen die Werte aller Knoten des Teilgraphen höchstens $\omega_i$ auseinander. Anders ausgedrückt
\begin{equation*}
\delta \geq \card{\left\lbrace \sum_{k=1}^p \left(\max_n a_{nk} - \min_n a_{nk}\right) \leq \omega_{k} \right\rbrace}\text{\@.}
\end{equation*}
\end{itemize}

\section{Wahl der Programmiersprache}

Imperativ: Gefahr unerwünschter wechselseitiger Beeinflussung, Gefahr Verklemmung bei Kommunikation\\
Funktional: Garantiert keine Nebenwirkung\\
Haskell: Pakete zur einfachen Parallelisierung, kaum Änderung des sequentiellen Codes nötig\\
zweischneidiges Schwert: Lazy-Evaluation $\rightarrow$ 1) nicht zu viele „Chunks” auf einmal 2) baut große Datenstrukturen, anstatt direkt zu reduzieren, wo das Gesamtergebnis immer benötigt wird (keine Parallelisierungs-Problem, muss man sich aber mit auseinandersetzen)


\section{Der Algorithmus}

Von Pseudocode rüberkopieren, eventuell anpassen an Details der Programmierung zur besseren Effizienz.



\section{Ausführung und Auswertung}
Amdahls Gesetz, Minskys Vermutung\\
Nach jedem Erweiterungsschritt: Sammeln und Aufgaben neu verteilen $\rightarrow$ Kommunikation

\section{Fazit}

\newpage
\printbibliography[heading=bibintoc]

\end{document}