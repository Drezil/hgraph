\documentclass[a4paper]{scrartcl}
\usepackage{titling}

% LANGUAGE and LOCALE
\usepackage[USenglish,ngerman]{babel} % Deutsches Wörterbuch usw.
\newcommand{\en}[1]{\selectlanguage{USenglish}#1\selectlanguage{ngerman}}

% FONT
\usepackage[T1]{fontenc}
\usepackage[utf8]{inputenc}
\usepackage[babel=true]{microtype} % after babel
\usepackage{lmodern}
%\usepackage{times}
\makeatletter
\g@addto@macro\@verbatim{\microtypesetup{activate=false}}
\makeatother

% ENUMERATE and ITEMIZE
\usepackage{enumerate}
\usepackage{paralist}

% SCIENCE and MATH
\usepackage{amsmath}
\usepackage{amssymb}
\usepackage{icomma}
%\usepackage{units}
\usepackage{xfrac} % Brüche im Stil von ½ (\sfrac)

% PROGRAMMING
\usepackage{algorithm}
\usepackage[noend]{algpseudocode}

% TABLES
%\usepackage{array}
%\usepackage{multicol}	
%\usepackage{dcolumn}
%\usepackage{slashbox} % diagonale Linie in Tabellenzelle
%\usepackage{tabularx}

% LINKS and REFS
\usepackage[obeyspaces]{url}
\usepackage{hyperref}

% BIBLIOGRAPHY
\usepackage[style=german,german=quotes]{csquotes}
\usepackage[style=alphabetic,backend=biber,babel=hyphen,maxbibnames=10]{biblatex}
%\usepackage[authordate,backend=biber,babel=hyphen]{biblatex-chicago} 
\bibliography{hgraph.bib}

% FORMATING
\usepackage{anysize} % Makros zum Einstellen der Seitenränder
\parindent0em % Keine amerikanische Einrückung am Anfang von Paragraphen
\clubpenalty = 10000  % Schusterungen bestrafen
\widowpenalty = 10000 % Hurenkinder bestrafen
\displaywidowpenalty = 10000

% math commands
\newcommand{\trans}[1]{{#1}^\intercal}
\newcommand{\abs}[1]{\left\vert #1 \right\vert}
\newcommand{\card}[1]{\left\vert #1 \right\vert} % cardinality
\newcommand{\norm}[1]{\left\Vert #1 \right\Vert}

\newcommand{\condset}[2]{\ensuremath{\left\lbrace #1\vphantom{#2}\right.\left\vert\; #2 \vphantom{#1}\right\rbrace}}

\newcommand{\setR}{\mathbb{R}}
\newcommand{\setN}{\mathbb{N}}

\DeclareMathOperator*{\argmin}{\arg\min}
\DeclareMathOperator*{\argmax}{\arg\max}

\hyphenation{IEEE} % hyphenation of default language
\begin{hyphenrules}{USenglish} % hyphenation of secondary language
  \hyphenation{IEEE}
\end{hyphenrules}

\begin{document}

\title{\en{Densely Connected Biclusters}}
\author{Stefan Dresselhaus \and Thomas Pajenkamp}
\date{\today} % TODO: Abgabedatum
\newcommand{\thesemester}{Wintersemester 2013/14}
\newcommand{\thecategory}{Programmierprojekt}
\newcommand{\thecourse}{Parallele Algorithmen und Datenverarbeitung}


%
% Deckblatt
%
\begin{titlepage}
  \begin{center} 
    \mbox{}
     \vspace{1cm}\\
     {\Huge \textbf{\thetitle}} \\[3em]
     {\huge \theauthor} \\[4em]
     {\Large \thecourse}\\[1em]
     {\Large \thecategory{} im \thesemester}\\[1em]
     
     \vspace{3cm}
     
     {\LARGE Universität Bielefeld -- Technische Fakultät}\\[2em]
     {\large \thedate}
  \end{center}
\end{titlepage}
%
% Inhaltsverzeichnis
%
%\setcounter{page}{1}
\microtypesetup{protrusion=false}
\tableofcontents
\microtypesetup{protrusion=true}

%\newpage

\section{Zielsetzung des Projekts}

Im Rahmen dieses Programmierprojekts wurde ein Programm entworfen und entwickelt, um \en{Densely Connected Biclusters}, im weiteren DCB, in einem biologischen Netzwerk zu ermitteln. Bei DCB handelt es sich um Teilgraphen eines Netzwerks, dessen Knoten untereinander hoch vernetzt sind und Objekte mit ähnlichen Eigenschaften repräsentieren. \par
Die Suche nach DCB ist ein NP-schweres Problem. Da mit einem geeigneten Algorithmus jedoch voneinander unabhängige Lösungspfade einzeln verfolgt werden können, ist das Problem gut für eine parallele Berechnung geeignet, wodurch die Gesamtlaufzeit stark reduziert werden kann. Zudem sind  real verwendete biologische Netze üblicherweise nur schwach vernetzt. Daher können durch die Forderung nach einer hohen Konnektivität der DCB viele Lösungskandidaten schnell ausgeschlossen werden und die schlimmstenfalls nichtdeterministisch polynomielle Laufzeit findet kaum Anwendung. \par

%TODO ein bisschen biologische Motivation?
% ... Sind wir Biologen? :p

\en{
\subsection{Densely Connected Biclusters}
}

Ausgangsbasis ist ein ungerichteter ungewichteter Graph $G = (V, E)$, dessen Knoten $n \in V$ mit jeweils $p$ Attributen versehen sind. Jedem Knoten $n$ ist zu jedem Attribut $i$ ein numerischer Wert $a_{ni}$ zugewiesen. \par
Ein DCB $D_k = (V_k, E_k)$ ist ein Teilgraph von $G$, der durch die Parameter $\alpha \in [0, 1]$, $\delta \in \setN$ und $\omega \in \setR^p$ beschränkt wird und die folgende drei Eigenschaften erfüllt.
\begin{itemize}
\item Der Teilgraph ist zusammenhängend.
\item Die Dichte des Teilgraphen unterschreitet einen Schwellenwert $\alpha$ nicht, also $\frac{2 \cdot \card{E_k}}{\card{V_k}(\card{V_k}-1)} \geq \alpha$.
\item Für mindestens $\delta$ Attribute liegen die Werte aller Knoten des Teilgraphen höchstens $\omega_i$ auseinander. Anders ausgedrückt
\begin{equation*}
\delta \leq \card{\condset{1\leq i \leq p}{\omega_k \geq \left(\max_{n\in V_k} a_{ni} - \min_{n\in V_k} a_{ni}\right)}} \text{\@.}
\end{equation*}
\end{itemize}


\section{Wahl der Programmiersprache und Konzept der Parallelisierung}

Die Wahl der Programmiersprache zur Verwirklichung des Projekts beeinflusst stark die Methoden der Programmierung und die Art der Parallelisierung. Klassischerweise werden für sequentielle und parallele Programme gleichermaßen imperative Sprachen wie C(++), Fortran oder auch Java verwendet, wofür Erweiterungen zur parallelen Programmierung existieren oder einige Werkzeuge direkt in die Sprache eingebaut sind. Bekannte Ansätze hierfür sind MPI und openMP. \par
Unser Projekt geht in eine etwas andere Richtung. Bei imperativer Programmierung muss ein großes Augenmerk auf die Vermeidung unerwünschter wechselseitiger Beeinflussungen verschiedener Threads und Prozesse gelegt werden, die fehlerhafte Rechenergebnisse zur Folge haben. Außerdem muss bei der Thread-/Prozesskommunikation immer die Gefahr von Verklemmungen beachtet werden, die schlimmstenfalls zu einem kompletten Stillstand der Programmausführung führen. Beide Probleme sind schwierig zu detektieren und zu lokalisieren.\par
Die genannten klassischen Probleme des Parallelrechnens können mit pur funktionaler Programmierung gut vermieden werden. Nebenbedingungen treten in pur funktionalem Programmcode (einen korrekten Compiler/Interpreter vorausgesetzt) garantiert nicht auf. Da das DCB-Problem bis auf das Einlesen der Eingabedaten und die Ausgabe pur funktional realisierbar ist, ist es optimal für eine derartige Implementierung geeignet Die konkrete Wahl der funktionalen Programmiersprache fiel auf \emph{Haskell}. \par
\medskip
Für Haskell wurden Bibliotheken entwickelt, die eine einfache und effiziente Programmierung paralleler Programme erlauben. Wir verwenden das Paket \emph{parallel} in Verbindung mit \emph{repa}-Arrays. Durch \emph{parallel} können geeignete Algorithmen mit wenig Aufwand, aufgeteilt werden. Diese Funktionsaufrufe werden unevaluiert in ein Array gepackt und dort von freien Threads abgearbeitet. Diese Technik nennt man Work-Stealing und die noch nicht ausgewerteten Funktionen werden in Haskell \emph{Sparks} genannt. Man kann sich dies als einen auf den Funktionsaufruf beschränkten light-weight Thread vorstellen - mit weniger Overhead. Die \emph{repa}-Arrays bieten Funktionen, um die einzelnen Arrayelemente parallel zu berechnen. Mit diesen Techniken lässt sich sequentieller Programmcode einfach parallelisieren, da hierfür nur wenige Änderungen erforderlich sind. Es müssen lediglich die Berechnungsfunktionen an die parallelisierende Funktion übergeben und die Funktion zur Auswertung der Arrayelemente ausgetauscht werden. \par
\medskip
Zwei wichtige Punkte müssen dennoch beachtet werden. Zum einen verwendet Haskell das Konzept \emph{\en{Lazy Evaluation}}. Befehle werden immer nur soweit berechnet, wie sie an anderer Stelle benötigt werden. Dadurch entstehen manchmal zur Laufzeit große Bäume nur teilweise ausgewerteter Befehle, welche die Ausführungszeit durch eine hohe Garbage-Collector-Auslastung stark negativ beeinflussen. Es muss demnach darauf geachtet werden, die Berechnung später ohnehin erforderlicher Funktionen frühzeitig zu erzwingen. Zum anderen ist die Anzahl der Sparks standardmäßig nicht begrenzt, sodass auch hier zu große Arrays entstehen können, deren Abarbeitung allerdings im Verlaufe des Programms durch o.g. Lazy Evaluation evtl. gar nicht erforderlich ist. Daher beschränken wir die Anzahl der möglichen Sparks (und somit der maximal Möglichen Worker-Threads) auf 1000. Erwähnenswert ist noch, dass diese Technik \emph{nicht} von Hyper-Threading profitiert (da nichtmal mehr ein Kontextwechsel der Threads nötig ist) und wir somit 1000 \glqq echte\grqq \ Kerne für eine maximale Auslastung benötigen. Die obere Grenze wird hier dann eher durch Amdahls Gesetz, denn durch die verfügbaren Kerne beschränkt.


\section{Der Algorithmus}

%TODO
An bereits erstelltem Pseudocode orientieren, eventuell anpassen an Details der Programmierung zur besseren Effizienz.



\section{Ausführung und Auswertung}
%TODO
Amdahls Gesetz, Minskys Vermutung\\
Nach jedem Erweiterungsschritt: Sammeln und Aufgaben neu verteilen $\rightarrow$ Kommunikation

\section{Fazit}
%TODO
Wir sind toll.

\newpage
\printbibliography[heading=bibintoc]

\end{document}